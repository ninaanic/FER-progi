\chapter{Zaključak i budući rad}
		
		 
		 Zadani zadatak naše grupe bio je razvoj rješenja problema vođenja inventure unutar više skladišta jedne kompanije, statistike greške skladištara koji vode inventuru i pomoću lokacije praćenje zaposlenika te kojim skladištima pripadaju. U razdoblju ovog semestra smo uspješno razvili aplikaciju te ostvarili zadani cilj. Provedba samog projekta odvijala se u dvije faze te smo unutar tima bili podijeljeni u dvije manje grupe zadužene za \textit{frontend} i \textit{backend} (podjela nije bila isključujuća te su članovi tima si međusobno kontinuirano pomagali radi bolje izvedbe i lakše komunikacije).
		 	
		 Prva faza razvoja aplikacije bila je okupljanje tima, dodjelu projektnog zadatka te praćenjem i dokumentiranjem zahtjeva određen obujam posla te inicijalna raspodjela poslova. Praćenjem daljnjih uputa te izradom obrazaca uporabe i dijagrama (sekvencijski dijagrami, dijagram razreda, model baze podataka (teže napravljen jer je baza podataka u Firebase-u, ali dovoljno da lakše shvatimo što je potrebno napraviti)) olakšana je raspodjela poslova među podtimovima i komunikacija među članovima jer smo se lakše mogli snaći u poslu koji je obavio drugi član. Izrada vizualnog dizajna aplikacije dodatno je olakšala razvoj u drugom ciklusu jer su podtimovi \textit{frontend} i \textit{backend} mogli brže raditi.
		 
		 Druga faza razvoja, programiranje zadane aplikacije, bila je puno zahtjevnija individualno zbog manjka iskustva u programiranju u zadanom razvojnom okruženju. Prije samog programiranja svaki član samostalno je proučio potrebne alate kako bi otprilike znali kako i izvedivost zahtjeva. Podjela poslova, iako u početku raspodijeljena individualno, većinom je tražila rad više članova koji su si međusobno pomagali i zajedno nalazili potrebna rješenja. Uz sam razvoj aplikacije bilo je potrebno daljnje praćenje UML dijagrama i održavanja projektne dokumentacije kako bi budući korisnici, a i članovi tima lakše snalazili u određenim dijelovima projekta. Kontinuirano praćenje i razvoja kostura aplikacije, prilagodba već napravljenih rješenja problema i intenzivna komunikacija članova omogućila nam je razvoj s minimalnim brojem grešaka i uštedu vremena u daljnjem programiranju.
		 
		 Za komunikaciju među članovima korištene su primarno dvije platforme: Whatsapp i Discord. Postojeće rješenje moguće je nastaviti razvijati i dalje koristeći mobilnu aplikaciju te čak i proširiti funkcionalnost dodatnim razvojem web aplikacije.
		 
		 Sudjelovanje na ovom projektu uvelike je koristilo svim članovima u razvoju vještina komuniciranja u većoj grupi, praćenje i pomaganje u radu više ljudi i učenjem u novim razvojnim okruženjima. Zadovoljni smo postignutim rješenjem koje je moguće nastaviti razvijati te primijeniti u stvarnom svijetu.
		
		\eject

\begin{comment}
	\textbf{\textit{dio 2. revizije}}\\
	
	\textit{U ovom poglavlju potrebno je napisati osvrt na vrijeme izrade projektnog zadatka, koji su tehnički izazovi prepoznati, jesu li riješeni ili kako bi mogli biti riješeni, koja su znanja stečena pri izradi projekta, koja bi znanja bila posebno potrebna za brže i kvalitetnije ostvarenje projekta i koje bi bile perspektive za nastavak rada u projektnoj grupi.}
	
	\textit{Potrebno je točno popisati funkcionalnosti koje nisu implementirane u ostvarenoj aplikaciji.}
\end{comment}